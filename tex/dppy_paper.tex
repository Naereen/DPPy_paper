\documentclass[twoside,11pt]{article}

% Any additional packages needed should be included after jmlr2e.
% Note that jmlr2e.sty includes epsfig, amssymb, natbib and graphicx,
% and defines many common macros, such as 'proof' and 'example'.
%
% It also sets the bibliographystyle to plainnat; for more information on
% natbib citation styles, see the natbib documentation, a copy of which
% is archived at http://www.jmlr.org/format/natbib.pdf

\usepackage{jmlr2e}
% \usepackage{adjustbox}
% \usepackage{algorithm}
% \usepackage{algorithmic}
\usepackage{amsfonts,amsmath,amssymb}
\usepackage{array}
% \usepackage[affil-it]{authblk}
\usepackage{bm} % Bold math
% \usepackage[style=icml2017.sty]{biblatex}
% \addbibresource[]{biblio.bib}
% \usepackage{caption}
% \usepackage{cancel}
% \usepackage{subcaption}
% \usepackage{datatool}
\usepackage{dirtytalk}
\usepackage{dsfont}
% \usepackage{enumitem}
% \usepackage{eso-pic}
\usepackage{epsfig}
\usepackage{etoolbox}
% \usepackage{fancybox}
\usepackage{fontawesome} % For GitHub icon
% \usepackage[para]{footmisc} % Footnotes on the same line, but implies 	hyperfootnotes=false...
% \usepackage[a4paper]{geometry}
% \geometry{rmargin=4cm,vmargin=3cm}
\usepackage{graphicx}
\graphicspath{{./images/}}
\usepackage{hyperref}
\hypersetup{
  colorlinks=true,
  linkcolor=blue,
  urlcolor=gray,
  citecolor=blue
}
\usepackage[utf8x]{inputenc}
\usepackage{layout}
% \usepackage{mathtools}
\usepackage{minted}
% See https://github.com/SublimeText/LaTeXTools/issues/657
% - install pygments https://pypi.org/project/Pygments/
% - update LaTeXTools.sublime-settings file with
%   - path to pygmentize add --shell-escape flag
%   - ST -> Preferences -> Package Settings -> LaTexTools -> Settings - User
%   - OSX "texpath" : "$PATH:....:/Users/ggautier/anaconda/bin"
%   - In "builder_settings" field add "options" : "--shell-escape",
% 
% \usepackage{monster2e}
% \usepackage{multimedia}
\usepackage{pgffor} % To create math commands recursively in commands.tex
\usepackage{natbib}
% \usepackage{newclude}
% \usepackage{microtype}
% \usepackage{setspace}
% \usepackage[absolute,showboxes,overlay]{textpos}
% \usepackage{tikz}
% \usetikzlibrary{calc,fadings,mindmap,trees,arrows,calc,tikzmark,shapes}
% \tikzstyle{every picture}+=[remember picture]
% \usepackage[colorinlistoftodos, textwidth=35mm, shadow]{todonotes}
% \usepackage{ulem}
% \usepackage{wasysym}
% \usepackage{wrapfig}
\usepackage{xcolor}
% Specific to this .tex
\newcommand{\DPPy}{\textsf{DPPy}}

\makeatletter
\@addtoreset{footnote}{page}
\makeatother
\renewcommand{\thefootnote}{\ifcase\value{footnote}\or\color{black}{\faGithub}\or\color{black}{\faBook}\fi} % For fancy footnotes

% Referencing
\newcommand*{\Chapref}[1]{Chapter~\ref{#1}}
\newcommand*{\Secref}[1]{Section~\ref{#1}}
\newcommand*{\Defref}[1]{Definition~\ref{#1}}
\newcommand*{\Thref}[1]{Theorem~\ref{#1}}
\newcommand*{\Propref}[1]{Proposition~\ref{#1}}
\newcommand*{\Lemref}[1]{Lemma~\ref{#1}}
\newcommand*{\Eqref}[1]{Equation~\ref{#1}}
\newcommand*{\Figref}[1]{Figure~\ref{#1}}

% Proof ends with filled box 
% \newcommand{\BlackBox}{\rule{1.5ex}{1.5ex}}  % end of proof
% \newenvironment{proof}{\par\noindent{\bf Proof\ }}{\hfill\BlackBox\\[2mm]}

%%%%%%%%%
% MATHS %
%%%%%%%%%

\renewcommand{\hat}{\widehat}
\renewcommand{\tilde}{\widetilde}
\newcommand{\minus}{\scalebox{0.9}{-}}
\newcommand{\plus}{\scalebox{0.6}{+}}
\newcommand{\equal}{\scalebox{0.6}{=}}

% Sum/Product with limits
\newcommand{\suml}{ \sum\limits }
\newcommand{\prodl}{ \prod\limits }

\newcommand{\minl}{\min \limits}
\def\minsub#1{\minl_{\substack{#1}}}
\newcommand{\maxl}{\max \limits}
\def\maxsub#1{\maxl_{\substack{#1}}}

\newcommand{\argmin}{\operatornamewithlimits{argmin}}
\newcommand{\argmax}{\operatornamewithlimits{argmax}}

% Reference measure
\newcommand\muref{\mu_{\text{ref}}}
% Reference density
\newcommand\fref{f_{\text{ref}}}
% Moment Problem
\newcommand\TMP{\operatorname{TMP}}
% Sum of Squares polynomial
\newcommand\SOS{\operatorname{SOS}}
% Moment sequence
\newcommand\bmom{\bm{m}}
% Canonical moment sequence
\newcommand\bc{\bm{c}}


%%% bold, cal and bb Uppercase letters A..Z
\foreach \x in {A,...,Z}
	{%
	\expandafter\xdef\csname bf\x \endcsname{\noexpand\ensuremath{\noexpand\bm{\x}}}
	\expandafter\xdef\csname cal\x \endcsname{\noexpand\ensuremath{\noexpand\mathcal{\x}}}
	\expandafter\xdef\csname bb\x \endcsname{\noexpand\ensuremath{\noexpand\mathbb{\x}}}
}

%%% bold greek letters
\foreach \x/\y  in {alpha/\alpha, beta/\beta, gamma/\gamma, delta/\delta, epsilon/\epsilon, zeta/\zeta, eta/\eta, theta/\theta, iota/\iota, kappa/\kappa, lambda/\lambda, mu/\mu, nu/\nu, pi/\pi, rho/\rho, sigma/\sigma, tau/\tau, upsilon/\upsilon, phi/\phi, chi/\chi, psi/\psi, omega/\omega }
	{%
\expandafter\xdef\csname bf\x \endcsname{\noexpand\ensuremath{\noexpand\bm{\y}}}
}

%% Equal = with limits
\def\xequal#1#2{\operatornamewithlimits{=}_{\substack{#1}}^{\substack{#2}}}
%% Equiv
\def\xequiv#1#2{\operatornamewithlimits{\Longleftrightarrow}_{\substack{#1}}^{\substack{#2}}}

%% SW--NE dots
\newcommand\rddots{\reflectbox{$\ddots$}}

% DPPs
% Support
\newcommand\supp{\operatorname{supp}}
% dist
\newcommand\dist{\operatorname{dist}}
% Convex Hull
\newcommand\conv{\operatorname{conv}}
% Configuration
\newcommand\Conf{\operatorname{Conf}}
% DPP
\newcommand\DPP{\operatorname{DPP}}
% Vandermonde
\newcommand\Vand{\operatorname{Vander}}
% Volume
\newcommand\Vol{\operatorname{Vol}}
% Zonotope
\newcommand\Zon{\calZ}
% Empirical measure
\newcommand\hmu{\hat{\mu}}
% z tilde
\newcommand\tz{\tilde{z}}


% % Itemize symbols
% \renewcommand{\labelitemi}{$\bullet$}
% \renewcommand{\labelitemii}{$\diamond$}
% \renewcommand{\labelitemiii}{$\ast$}
% \renewcommand{\labelitemiv}{$\cdot$}

% Notations
\newcommand\ie{\text{i.e.\ }}
\newcommand\eg{\text{e.g.\ }}
\newcommand\wrt{\text{w.r.t.\ }}
\newcommand\etc{\text{etc.\ }}
\newcommand\resp{\text{resp.\ }}
\newcommand\iid{\text{i.i.d.\ }}


% differential notation dx
\newcommand*\diff{\mathop{}\!\mathrm{d}}
\newcommand*\Diff[1]{\mathop{}\!\mathrm{d^#1}}
\renewcommand{\d}[1]{\ensuremath{\operatorname{d}\!{#1}}}
% exponential
\newcommand\e{\operatorname{\mathrm{e}}}
% erf
\newcommand\erf{\operatorname{erf}}
% complex number
\newcommand\ii{\bm{i}}

%%%%%%%%%%%%%%%%%%%%%%%
% Size adaptive symbols
%%%%%%%%%%%%%%%%%%%%%%%
% brackets
\newcommand{\lrb}[1]{\left[ #1 \right]}
% parenthesis
\newcommand{\lrp}[1]{\left( #1 \right)}
% curly brackets (typically sets)
\newcommand{\lrcb}[1]{\left\{ #1 \right\}}
\newcommand{\lrllb}[1]{\left\llbracket #1 \right\rrbracket}
% scalar product
\newcommand{\lrsp}[1]{\left\langle #1 \right\rangle}
% absolute value
\newcommand{\lrabs}[1]{\left| #1 \right|}
% norm
\newcommand{\lrnorm}[1]{\left\| #1 \right\|}
% floor
\newcommand{\floor}[1]{\left\lfloor #1 \right\rfloor}

%%%%%%%%%%%%%%%%%%%%%
% Probability symbols
%%%%%%%%%%%%%%%%%%%%%

\newcommand\Prob{\operatorname{\bbP}}
\newcommand\Esp{\operatorname{\bbE}}
\newcommand\Vari{\operatorname{\bbV ar}}
\newcommand\Cova{\operatorname{\bbC ov}}

% Adaptive brackets
\newcommand{\Proba}[1]{\Prob\lrb{#1}}
\newcommand{\Espe}[1]{\Esp\lrb{#1}}
\newcommand{\Var}[1]{\Vari\lrb{#1}}
\newcommand{\Cov}[1]{\Cova\lrb{#1}}

% Independent symbol similar to perp
\def\independenT#1#2{\mathrel{\rlap{$#1#2$}\mkern2mu{#1#2}}}
\newcommand\indep{\protect\mathpalette{\protect\independenT}{\perp}}
%%%%%%%%%%%%%%%
% Distributions

% Indicator function
\newcommand\indic{\mathds{1}}
% Bernoulli
\newcommand\Ber{\operatorname{\calB er}\!}
% Beta
\newcommand\Beta{\operatorname{Beta}}
% Beta
\newcommand\Dir{\operatorname{Dir}\!}
% Uniform
\newcommand\Unif{\operatorname{\calU}\!}
% Poisson
\newcommand\Poiss{\operatorname{\calP oiss}\!}
% Gaussian
\newcommand\Norm{\operatorname{\calN}\!}
%Chi 2
\newcommand{\ChiTwo}[1]{\mbox{\large$\chi_{#1}^2$}}

%%%%%%%%%%%%%
% Set symbols
%%%%%%%%%%%%%
% Complement
\newcommand\comp{\mathsf{c}}

% Permutations
\newcommand\Sn{\mathfrak{S}_n}

% Unit circle
\newcommand\Sone{\bbS^1}

%%%%%%%%%%
% Topology
%%%%%%%%%%
\newcommand\Int{\operatorname{Int}}

%%%%%%%%%%%%%%%%
% Linear Algebra
%%%%%%%%%%%%%%%%
\renewcommand{\top}{\mathsf{\scriptscriptstyle T}}

\newcommand\RE{\operatorname{Re}}
\newcommand\IM{\operatorname{Im}}
\newcommand\Ker{\operatorname{Ker}}
\newcommand\Span{\operatorname{Span}}

\newcommand\Tr{\operatorname{Tr}}
\newcommand\rank{\operatorname{rank}}
\newcommand\diag{\operatorname{diag}}
\newcommand\Sp{\operatorname{Sp}}


%%% citepos command for adding 's to author name
% e.g. \citepos{Wig67} yields Wigner's (1967)
\makeatletter
% make numeric styles use name format
\patchcmd{\NAT@test}{\else \NAT@nm}{\else \NAT@nmfmt{\NAT@nm}}{}{}
% define \citepos just like \citet
\DeclareRobustCommand\citepos
  {\begingroup
   \let\NAT@nmfmt\NAT@posfmt% ...except with a different name format
   \NAT@swafalse\let\NAT@ctype\z@\NAT@partrue
   \@ifstar{\NAT@fulltrue\NAT@citetp}{\NAT@fullfalse\NAT@citetp}}
\let\NAT@orig@nmfmt\NAT@nmfmt
\def\NAT@posfmt#1{\NAT@orig@nmfmt{#1's}}

% Heading arguments are {volume}{year}{pages}{date submitted}{date published}{paper id}{author-full-names}

\jmlrheading{xx}{2018}{xx-xx}{8/18}{xx/xx}{gaba18}{Guillaume Gautier and R\'emi Bardenet}

% Short headings should be running head and authors last names

\ShortHeadings{\DPPy}{Gautier, Bardenet, and Valko}
\firstpageno{1}

\begin{document}

\title{\DPPy: a Python Toolbox for Sampling\\Determinantal Point Processes}

\author{\name Guillaume Gautier \email guillaume.gga@gmail.com \\
       \name R\'emi Bardenet$^\dagger$ \email remi.bardenet@gmail.com \\
       \name Michal Valko \email remi.bardenet@gmail.com \\
       \addr $^\dagger$ Univ. Lille, CNRS, Centrale Lille, UMR 9189 - CRIStAL, 59651 Villeneuve d’Ascq, France
}

\editor{}

\maketitle

\setcounter{footnote}{3}
\begin{abstract}%   <- trailing '%' for backward compatibility of .sty file

  Determinantal point processes (DPPs) are specific probability distributions over clouds of points that have been popular as models or computational tools across physics, probability, statistics, and more recently of booming interest in machine learning.
  Sampling from DPPs is a nontrivial matter. \DPPy\ is a Python toolbox that puts together exact and approximate sampling algorithms from the literature.
  The project is hosted on GitHub\footnote{\url{https://github.com/guilgautier/DPPy}} and comes with an extensive documentation\footnote{\label{fn:docs}\url{https://dppy.readthedocs.io}}. The documentation takes the form of a short survey of DPPs, and relates each mathematical property with \DPPy\ objects for ease of manipulation.

\end{abstract}
% \setcounter{footnote}{0}

\begin{keywords}
  Determinantal Point Processes, Sampling Schemes.
\end{keywords}

\section{Introduction} % (fold)
\label{sec:introduction}

  Determinantal point processes (DPPs) are distributions over configurations of points that encode diversity through a kernel function $K$ and a reference measure $\mu$.
  There were introduced by \citet{Mac75} and have then found applications in fields as diverse as probability \citep{Sos00, Kon05, HKPV06}, number theory \citep{RuSa96}, statistical physics \citep{PaBe11}, Monte Carlo methods \citep{BaHa16}, and spatial statistics \citep{LaMoRu15}.
  In machine learning, DPPs over finite sets have been used as a model of diverse sets of items, where the kernel function takes the form of a finite matrix $\bfK$, see \citep{KuTa12} for a comprehensive survey.
  Further applications of DPPs in machine learning (ML) e.g. include recommendation tasks \citep{KaDeKo16, GaPaKo16} or text summarization \citep{DuBa16}.\\

  In the finite case, routine inference tasks such as normalization, marginalization and sampling have complexity roughly $\calO(M^3)$, where $M$ denotes the size of the ground set \citep{KuTa12}. This becomes intractable for large $M$ \citep{Gil14}. In particular, for generic kernel matrices, exact sampling requires the eigendecomposition of the kernel. This prohibitive cost has triggered an effort towards efficient approximate samplers ranging from random approximation \todo{Maybe find a different way of saying it, "random approximation" is too vague} of the kernel \citep{AfKuFo13} to MCMC samplers \citep{AnGhRe16, LiJeSr16c, GaBaVa17}.\\

  \DPPy\ is the first toolbox which provides a turnkey implementation of all general algorithms to sample DPPs. \citet{TrBaAm18} also recently surveyed the main exact sampling schemes, but without related code. To the best of our knowledge, there is scarcely any reproducible papers involving DPPs; sampling schemes mostly remain in pseudo-code form with no clean implementation. \todo{"To the best..." Too harsh. Rather say sth like "Packaging together all known sampling algorithms now and in the future will improve reproducibility of papers on DPPs and favour dissemination of results."}  \setcounter{footnote}{5}
    Relatedly, we use Travis\footnote{\url{https://travis-ci.com/guilgautier/DPPy}} for continuous integration, both to run tests and certify that \DPPy\ can be used with Linux and MacOS. Moreover, \DPPy\ is supported by an extensive documentation providing the essential mathematical background; this illustrates some key properties of DPPs through \DPPy\ objects and associated methods. \DPPy\ can thus also easily serve as tutorial material.



% section introduction (end)

\section{Determinantal Point Processes} % (fold)
\label{sec:determinantal_point_processes}

  Something to say that technicalities and subtleties are avoided on purpose...

  \subsection{Definition} % (fold)
  \label{sub:definition}

    A point process $\calX$ on $\bbX$, can be roughly viewed as a random subset of point $\lrcb{X_1, \dots, X_N} \subset \bbX$, with $N<\infty$ but a priori random.
    To understand the interaction between the points $\lrcb{X_1, \dots, X_N}$, one focuses on the interaction of each cloud of $k$ points (for all $k$).
    The corresponding $k$-correlation functions characterize the underlying point process.
    % Considering $\mu$ as the reference measure, the $k$-correlation functions are defined as
    % \begin{equation}
    % \label{eq:k-correlation_function}
    %   \Expe{ \sum_{
    %     \substack{
    %       (X_1,\dots,X_k) \\
    %       X_1 \neq \dots \neq X_k} }
    %     f(X_1,\dots,X_k)
    %     }
    %     = \int_{\bbX^k}
    %       f(x_1,\dots,x_k) \rho_k(x_1,\dots,x_k)
    %       \prod_{i=1}^k \mu(dx_i),
    % \end{equation}

    % for all bounded measurable functions $f:\bbX^k\to \bbC$ and the sum ranges over $k$-tuples of distinct points of $\calX$.
    A point process parametrized by a kernel $K$ associated to the reference measure $\mu$ is said to be determinantal and noted $\DPP(K)$ if its $k$-correlation functions read
    \begin{equation}
    \label{eq:k-correlation_function_DPP}
      \rho_k(x_1,\dots,x_k)
        = \det [K(x_i, x_j)]_{i,j=1}^k,
      \quad \forall k\geq 1.
    \end{equation}

    The finite case, when $\bbX = \lrcb{1,\dots,M}$ with $\mu=\sum_{i=1}^M \delta_i$, corresponds to most common use case of DPPs in ML \citep{KuTa12}.
    In this context, the kernel function becomes a $M\times M$ matrix $\bfK$ encoding the similarity between items, and the correlation functions refer simply to inclusion probabilities.
    We say that $\calX\sim \DPP(\bfK)$ if
    \begin{equation}
    \label{eq:inclusion_proba_finite}
      \Proba{S \subset \calX} = \det \bfK_S,
        \quad\forall S\subset \bbX,
    \end{equation}

    where $\bfK_S$ denotes the submatrix of $\bfK$ formed by the rows and columns indexed by $S$.

    To guarantee existence of the DPP as described by Equations~\ref{eq:k-correlation_function_DPP} and~\ref{eq:inclusion_proba_finite} the underlying kernel must have $[0,1]$-valued eigenvalues and it is commonly assumed to be Hermitian symmetric, see \citep[Theorem 3]{Sos00}.
    The latter assumption is only sufficient, there exist DPPs with non symmetric kernels, see \citep{BoDiFu09}.
  % subsection definition (end)

  \subsection{Sampling} % (fold)
  \label{sub:sampling}

    \subsubsection{Exact Sampling} % (fold)
    \label{ssub:exact_sampling}

      The determinantal feature confer DPPs many appealing properties, among which closed form expression for XXX, moments of linear statistics \etc
      For our sampling purposes there is one crucial property: generic DPPs are mixtures of \textit{projection} DPPs i.e. DPPs parametrized by an orthogonal projection kernel.
      Namely, given the spectral decomposition of the kernel
      \begin{equation}
      \label{eq:eigdec_kernel}
        K(x,y)=\suml_{i=1}^{\infty} \lambda_i \phi_i(x) \overline{\phi_i(y)},
        \quad \text{with }
        \int \phi_i(x) \overline{\phi_j(x)} \mu(\diff x) = \delta_{ij},
      \end{equation}

      define the random orthogonal projection kernel
      $
        K^B(x,y)
          = \suml_{i=1}^{\infty} B_i \phi_i(x) \overline{\phi_i(y)}
      $, with independent $B_i\sim\Ber(\lambda_i)$.
      \citet[Theorem 7]{HKPV06} give
      \begin{equation}
        \DPP(K) \sim \DPP(K^B),
      \end{equation}
      suggesting the following two steps algorithm to sample from $\DPP(K)$:
      \begin{enumerate}
        \item Draw $B_i\sim\Ber(\lambda_i)$ independently and note $\lrcb{i_1,\dots,i_{N}} = \lrcb{i~;~B_i=1}$.
        \item Sample from the \textit{projection} DPP with kernel $\tilde{K}(x,y) = \sum_{n=1}^{N}\phi_{i_n}(x) \overline{\phi_{i_n}(y)}$.
      \end{enumerate}

      The remaining question of sampling from \textit{projection} DPPs is addressed by \citet{HKPV06}[Algorithm 18].
      It based on the chain rule and the fact that \textit{projection} $\DPP(\tilde{K})$ generates configurations of $N=\Tr \tilde{K}$ points almost surely.
      In the first phase each point $x\in \bbX$ is associated to the random feature vector $\Phi(x)=\lrp{\phi_{i_1}(x),\dots,\phi_{i_N}(x)}$, therefore $\tilde{K}(x,y) = \Phi(y)^{\dagger} \Phi(x)$.
      % $N\overset{\text{a.s.}}{=}\Tr K$ points.
      In this setting, the joint distribution of $(X_1,\dots,X_N)$ reads
      \begin{equation}
      \label{eq:joint_distribution}
        \frac{1}{N!} \det \lrb{K(x_m,x_n)}_{m,n=1}^N \prod_{n=1}^N\mu(\diff x_n)
          = \frac{1}{N!} \Vol^2\lrcb{\Phi(x_1),\dots,\Phi(x_n)} \prod_{n=1}^N\mu(\diff x_n),
      \end{equation}

      so that the conditional densities appearing in the chain rule take the following form
      \begin{equation}
      \label{eq:conditionals_densities}
        g_1(x)
          = \frac{1}{N} \lrnorm{\phi(x)}^2
          \quad\text{and}\quad
        g_{n | 1:n-1}(x)
          = \frac{1}{N-(n-1)} \lrnorm{ \Pi_{H_{n-1}^{\perp}} \phi(x)}^2,
      \end{equation}
      % \begin{align*}
      %   g_1(x)           &= \frac{1}{N} \lrnorm{\phi(x)}^2
      %                    &= \frac{1}{N} K(x,x) \\
      %   g_{n | 1:n-1}(x) &= \frac{1}{N-(n-1)}
      %                       \lrnorm{ \Pi_{H_{n-1}^{\perp}} \phi(x)}^2 \\
      %                    &= \frac{1}{N-(n-1)}
      %                       \lrb{K(x,x) - \overline{K(x,x_{1:n-1})} \lrb{K(x_k,x_l)}_{k,l=1}^{n-1} K(x_{1:n-1},x)}
      % \end{align*}
      where $\Pi_{H_{n-1}}$ is the orthogonal projection onto
      $H_{n-1} = \Span\lrcb{\phi(x_1), \dots, \phi(x_{n-1}) }$.
      The exact sampling procedure has a strong Gram-Schmidt flavor, and the product of the conditional densities in \Eqref{eq:conditionals_densities} refers to the base$\times$height formula.
      In the end, DPPs favors configuration of points whose feature vectors $\Phi(x_1),\dots, \Phi(x_N)$ span a large volume.\\

      The previous sampling scheme is exact and generic but requires the eigendecomposition of the underlying kernel.
      In the finite setting, this corresponds to an initial $\calO(M^3)$ cost, but then the procedure has average cost $\calO(M\lrb{\Tr \bfK}^2)$, which may still be problematic for large $M$, see \citep{TrBaAm18}.

      On the other hand, sampling from the conditionals in \Eqref{eq:conditionals_densities} can be cumbersome in the infinite case, since it may require tailored proposal distributions.\\

      These issues have motivated work on approximate sampling procedures with two main directions: kernel approximation and MCMC samplers.
      It is worth mentioning that some specific DPPs admit special exact samplers, \eg uniform spanning trees \citep{PrWi98} or eigenvalues of some random matrices \citep{DuEd02}.

    % subsubsection exact_sampling (end)

    \subsubsection{Random Matrices} % (fold)
    \label{ssub:random_matrices}

      Originally, continuous DPPs emerged in quantum mechanics and mathematical physics with the pioneering work of \citet{Wig67} relating eigenvalues of random matrices to energy levels of heavy atoms.
      That is to say, diagonalizing a properly randomized $N\times N$ matrix is a way to generate exact samples of continuous DPPs at cost $\calO(N^3)$.
      More generally, $\beta$-Ensembles for which the case $\beta=2$ refers to a specific class of \textit{projection} DPPs \citep{Kon05} have joint probability distribution
      \begin{equation}
        \frac{1}{Z_{N,\beta}}
        \lrabs{\Delta(x_1,\dots,x_N)}^{\beta}
        \prod_{i= 1}^N
          \mu(d x_i),
      \end{equation}

      where $\lrabs{\Delta(x_1,\dots,x_N)}$ denotes the Vandermonde determinant, and $\beta$ plays the role of an inverse temperature controlling the strength of the repulsive interaction between the points.
      The cases $\beta=1, 2$ and $4$ together with Gaussian, Gamma and Beta reference measure $\mu$ received special attention, since they correspond to the distribution of some random matrix models.
      \citet{DuEd02} unlocked generic $\beta>0$ parameters through equivalent tridiagonal models, thus reducing the sampling cost to $\calO(N^2)$.

    % subsubsection random_matrices (end)

    \subsubsection{Approximate Sampling} % (fold)
    \label{ssub:approximate_sampling}

      Kernel approximation, MCMC, no room

    % subsubsection approximate_sampling (end)

  % subsection sampling (end)

% section determinantal_point_processes (end)

\section{The \DPPy\ Toolbox} % (fold)
\label{sec:the_dppy_toolbox}

  \DPPy\ strives to handle objects that fit the natural definition of the different DPPs models:
  \begin{itemize}
	  \item The \DPPy\ object corresponding to the finite $\DPP(\bfK)$ can be instantiated as
	  \begin{nscenter}
	  	\mintinline{python}{Finite_DPP(kernel_type="inclusion", projection=False, **{"K":K})}.
	  \end{nscenter}
		It has two main sampling methods namely \mintinline{python}{.sample_exact()} and \mintinline{python}{.sample_mcmc()} implementing different variants of the exact sampling scheme and current state of the art MCMC samplers.

		\item The \DPPy\ object corresponding to $\beta$-Ensembles can be instantiated as
		\begin{nscenter}
			\mintinline{python}{Beta_Ensemble(ensemble_name="laguerre", beta=3.14)},
		\end{nscenter}
		when $\mu$ is Gamma (parametrized using the \mintinline{python}{numpy} style).
		It has one sampling method
		\begin{nscenter}
			\mintinline{python}{.sample(sampling_mode="banded", **{"shape":10, "scale":2.0, "size":50})}
		\end{nscenter}
		and two methods for display; \mintinline{python}{.plot()} to plot the last realisation and \mintinline{python}{.hist()} to construct the empirical distribution.

  \end{itemize}

  For more information, please refer to the documentation\footnoteref{fn:docs} and corresponding notebooks that showcase the use \DPPy\ objects.


% section the_dppy_toolbox (end)


\section{Conclusion and Future Work} % (fold)
\label{sec:conclusion_and_future_work}

The release of \DPPy\ marks the first reproducible and fully documented entry point for using DPPs. It is ready to raise contributors' interest to consolidate the existing content and enlarge the scope of \DPPy\ \eg with procedures for learning the kernel from data.

% section conclusion_and_future_work (end)

% Acknowledgements should go at the end, before appendices and references
\newpage
\acks{The research presented was supported by French Ministry of Higher Education and Research, CPER Nord-Pas de Calais/FEDER DATA Advanced data science and tech- nologies 2015-2020, and French National Research Agency projects EXTRA-LEARN (n.ANR-14-CE24-0010-01) and BOB (n.ANR-16-CE23-0003).}


\section{Links to JMLR } % (fold)
\label{sec:links_to_jmlr_}

% section links_to_jmlr_ (end)
  \begin{itemize}
    \item \href{http://www.jmlr.org/mloss/mloss-info.html}{MLOSS website}
    \item \href{http://jmlr.csail.mit.edu/author-info.html}{JMLR author info}
    \item \href{http://www.jmlr.org/format/authors-guide.html}{JMLR author guide}
  \end{itemize}

% Manual newpage inserted to improve layout of sample file - not
% needed in general before appendices/bibliography.

% \newpage

% \appendix
% \section*{Appendix A.}
% \label{app:theorem}

% Note: in this sample, the section number is hard-coded in. Following
% proper LaTeX conventions, it should properly be coded as a reference:

%In this appendix we prove the following theorem from
%Section~\ref{sec:textree-generalization}:

% \vskip 0.2in

\bibliographystyle{icml2017}
\bibliography{biblio}

\end{document}
