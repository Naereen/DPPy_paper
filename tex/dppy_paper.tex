\documentclass[twoside,11pt]{article}

% Any additional packages needed should be included after jmlr2e.
% Note that jmlr2e.sty includes epsfig, amssymb, natbib and graphicx,
% and defines many common macros, such as 'proof' and 'example'.
%
% It also sets the bibliographystyle to plainnat; for more information on
% natbib citation styles, see the natbib documentation, a copy of which
% is archived at http://www.jmlr.org/format/natbib.pdf

\usepackage{jmlr2e}

%%% Packages
\usepackage{amsfonts,amsmath,amssymb}
\usepackage{fontawesome} % For fancy icons :)
\usepackage{hyperref}
\hypersetup{
	breaklinks=true,
  colorlinks=true,
  linkcolor=blue,
  urlcolor=gray,
  citecolor=blue
}
\usepackage[utf8x]{inputenc}
\usepackage{pgffor} % To create math commands recursively in commands.tex
\usepackage{natbib}
\usepackage{xcolor}
%%% end packages

%%% Commands

% Specific to this .tex
% DPP
\newcommand{\DPPy}{\textsf{DPPy}}
\newcommand\DPP{\operatorname{DPP}}

% Python code highlight
\definecolor{deepblue}{rgb}{0,0,0.5}
\definecolor{deepred}{rgb}{0.8,0,0}
\definecolor{deepgreen}{rgb}{0,0.5,0}

\newcommand{\pygreen}[1]{\textcolor{deepgreen}{#1}}
\newcommand{\pystr}[1]{\textcolor{deepred}{"#1"}}
\newcommand{\pylrcb}[1]{\string{#1\string}}
\newcommand{\pykwargs}{\textcolor{magenta}{**}}
\newcommand{\pyeq}{\textcolor{magenta}{=}}
\newcommand{\pyus}{\texttt{\char`_}}

% TODO command
\newcommand*{\todo}[1]{\textcolor{red}{TODO:{#1}}}
\newcommand*{\anstodo}[1]{\textcolor{blue}{ANS-TODO:{#1}}}

% Notations
\newcommand\eg{\text{e.g., }}
\newcommand\iid{\text{i.i.d.\,}}

% Footnotes 1, 2, 3 and then fancy ones!
\makeatletter
\@addtoreset{footnote}{page}
\makeatother

\renewcommand{\thefootnote}{\ifcase\value{footnote}\or\color{black}{1}\or\color{black}{2}\or\color{black}{3}\or\color{black}{\faGithub}\or\color{black}{\faBook}\or\color{black}{\faGears}\fi} 

% Cross ref a foonote
\makeatletter
\newcommand\footnoteref[1]{\protected@xdef\@thefnmark{\ref{#1}}\@footnotemark}
\makeatother

\addtolength{\skip\footins}{-3pc plus 5pt}

% Center env for minted package to display code
\newenvironment{nscenter}
 {\parskip=3pt\par\nopagebreak\centering}
 {\parskip=2pt\par\noindent\ignorespacesafterend}

\newcommand*{\Eqref}[1]{Equation~\ref{#1}}

%%%%%%%%%
% MATHS %
%%%%%%%%%

\renewcommand{\tilde}{\widetilde}

% Sum/Product with limits
\newcommand{\suml}{ \sum\limits }
\newcommand{\prodl}{ \prod\limits }

%%% bold, cal and bb Uppercase letters A..Z
\foreach \x in {A,...,Z}
	{%
	\expandafter\xdef\csname bf\x \endcsname{\noexpand\ensuremath{\noexpand\bf{\x}}}
	\expandafter\xdef\csname cal\x \endcsname{\noexpand\ensuremath{\noexpand\mathcal{\x}}}
	\expandafter\xdef\csname bb\x \endcsname{\noexpand\ensuremath{\noexpand\mathbb{\x}}}
}

\newcommand\Vol{\operatorname{Vol}}

% differential notation dx
\newcommand*\diff{\mathop{}\!\mathrm{d}}

%%%%%%%%%%%%%%%%%%%%%%%
% Size adaptive symbols
%%%%%%%%%%%%%%%%%%%%%%%
% brackets
\newcommand{\lrb}[1]{\left[ #1 \right]}
% parenthesis
\newcommand{\lrp}[1]{\left( #1 \right)}
% curly brackets (typically sets)
\newcommand{\lrcb}[1]{\left\{ #1 \right\}}
% absolute value
\newcommand{\lrabs}[1]{\left| #1 \right|}
% norm
\newcommand{\lrnorm}[1]{\left\| #1 \right\|}

%%%%%%%%%%%%%%%%%%%%%
% Probability symbols
%%%%%%%%%%%%%%%%%%%%%

\newcommand\Prob{\operatorname{\bbP}}
\newcommand\Exp{\operatorname{\bbE}}

% Adaptive brackets
\newcommand{\Proba}[1]{\Prob\lrb{#1}}
\newcommand{\Expe}[1]{\Exp\lrb{#1}}

% Independent symbol similar to perp
\def\independenT#1#2{\mathrel{\rlap{$#1#2$}\mkern2mu{#1#2}}}
\newcommand\indep{\protect\mathpalette{\protect\independenT}{\perp}}

% Distributions
\newcommand\Ber{\operatorname{\calB er }}

%%%%%%%%%%%%%%%%
% Linear Algebra
%%%%%%%%%%%%%%%%
\renewcommand{\top}{\mathsf{\scriptscriptstyle T}}

\newcommand\Span{\operatorname{span}}

\newcommand\Tr{\operatorname{Tr}}
\newcommand\rank{\operatorname{rank}}

%%% end commands

% Heading arguments are {volume}{year}{pages}{date submitted}{date published}{paper id}{author-full-names}

\jmlrheading{xx}{2018}{xx-xx}{9/18}{xx/xx}{gabava18}{Guillaume Gautier, R\'emi Bardenet, and Michal Valko}

% Short headings should be running head and authors last names

\ShortHeadings{\DPPy}{Gautier, Bardenet, and Valko}
\firstpageno{1}

\begin{document}

\title{\DPPy: Sampling Determinantal Point Processes with Python}

\author{\name Guillaume Gautier$^{*,\dagger}$ \email g.gautier@inria.fr \\
       \name R\'emi Bardenet$^\dagger$ \email remi.bardenet@gmail.com \\
       \name Michal Valko$^*$ \email michal.valko@inria.fr\\
       \addr $^*$SequeL team, INRIA Lille - Nord Europe,  40, avenue Halley 59650, Villeneuve d'Ascq, France\\
       \addr $^\dagger$Univ.\,Lille, CNRS, Centrale Lille, UMR 9189 - CRIStAL, 59651 Villeneuve d'Ascq, France
}

\editor{}

\maketitle

\setcounter{footnote}{3}
\begin{abstract}%   <- trailing '%' for backward compatibility of .sty file
  Determinantal point processes (DPPs) are specific probability distributions over clouds of points that are used as models and computational tools across physics, probability, statistics, and more recently machine learning.
  Sampling from DPPs is a challenge and therefore we present \DPPy, a Python toolbox that gathers known exact and approximate sampling algorithms.
  The project is hosted on GitHub\footnote{\label{fn:github}\url{https://github.com/guilgautier/DPPy}} and equipped with an extensive documentation.
  This documentation\footnote{\label{fn:docs}\url{https://dppy.readthedocs.io}} takes the form of a short survey of DPPs and relates each mathematical property with \DPPy\ objects.
\end{abstract}

\begin{keywords}%
  determinantal point processes,
  sampling schemes,
  random matrices
\end{keywords}

\section{Introduction} % (fold)
\label{sec:introduction}

Determinantal point processes (DPPs) are distributions over configurations of points that encode diversity through a kernel function $K$.
They were introduced by \citet{Mac75} as models for beams of fermions, and they have since found applications in fields as diverse as probability \citep{Sos00, Kon05, HKPV06}, statistical physics \citep{PaBe11}, Monte Carlo methods \citep{BaHa16}, spatial statistics \citep{LaMoRu15}, and machine learning \citep[ML,][]{KuTa12}.

In ML, DPPs mainly serve to model diverse sets of items, as in recommendation \citep{KaDeKo16, GaPaKo16} or text summarization \citep{DuBa18}.
Consequently, MLers  use mostly finite DPPs, which are distributions over subsets of a finite \emph{ground set} of cardinality $M$, parametrized by an $M\times M$ kernel matrix $\bfK$.
Routine inference tasks such as normalization, marginalization, or sampling have complexity $\calO(M^3)$ \citep{KuTa12}.
For large $M$, $\calO(M^3)$ is a bottleneck, as for other kernel methods, see also \citet{TrBaAm18} for a survey on exact sampling.
Efficient approximate samplers have thus been developed, ranging from kernel approximation \citep{AKFT13} to MCMC samplers \citep{AnGhRe16, LiJeSr16c, GaBaVa17}.

In terms of software, the R library \textsf{spatstat}\ \citep{BaTu05}, a general-purpose toolbox on spatial point processes, includes sampling and learning of continuous DPPs with stationary kernels, as described by \citet{LaMoRu15}.
On the other hand, we propose \DPPy, a turnkey implementation of all known general algorithms to sample \emph{finite} DPPs.
We also provide a few algorithms for non-stationary continuous DPPs that are related to random projections and random covariance matrices, which can be of interest for MLers.

\DPPy\ is hosted on GitHub.\footnoteref{fn:github} We use \setcounter{footnote}{5}Travis\footnote{\url{https://travis-ci.com/guilgautier/DPPy}} for continuous integration.
%, both to run tests and certify that \DPPy\ can be used with Linux and MacOS.
Moreover, \DPPy\ is supported by an extensive documentation\footnoteref{fn:docs} which provides the essential mathematical background and illustrates some key properties of DPPs through \DPPy\ objects and their methods.
\DPPy\ thus also serves as a tutorial.
Along the paper, words in magenta point to the documentation.

% section introduction (end)

\section{Determinantal point processes} % (fold)
\label{sec:determinantal_point_processes}

	We introduce DPPs and the main sampling algorithm; see \citet{HKPV06} for details.

	\subsection{Definition} % (fold)
	\label{sub:definition}

	A point process $\calX$ on $\bbX$ is a random subset of points $\lrcb{X_1, \dots, X_N} \subset \bbX$, where the number of points $N$ is itself random.
	We further add to the definition that $N$ should be almost surely finite and that all points in a sample are distinct.
	Given a reference measure $\mu$ on~$\bbX$, a point process is usually characterized by its $k$-correlation function $\rho_k$ for all $k$, where
	\begin{equation*}
	\label{eq:correlation_function_intuition}
		\Proba{
			\begin{tabular}{c}
				$\exists$ one point of the process in\\
				each ball $B(x_i, \diff x_i), \forall i=1,\dots, k $
			\end{tabular}
		}
		= \rho_k\lrp{x_1,\dots,x_k}
			\prod_{i=1}^k \mu(\diff x_i),
	\end{equation*}
	see \citet[Section 4]{MoWa04}.
	The functions $\rho_k$ describe the interaction among points in $\calX$ by quantifying co-occurrence of points at a set of locations.
	% Considering $\mu$ as the reference measure, the $k$-correlation functions are defined as
	% \begin{equation}
	% \label{eq:k-correlation_function}
	%   \Expe{ \sum_{
	%     \substack{
	%       (X_1,\dots,X_k) \\
	%       X_1 \neq \dots \neq X_k} }
	%     f(X_1,\dots,X_k)
	%     }
	%     = \int_{\bbX^k}
	%       f(x_1,\dots,x_k) \rho_k(x_1,\dots,x_k)
	%       \prod_{i=1}^k \mu(dx_i),
	% \end{equation}

	% for all bounded measurable functions $f:\bbX^k\to \bbC$ and the sum ranges over $k$-tuples of distinct points of $\calX$.

	A point process $\calX$ on $(\bbX,\mu)$ parametrized by a kernel $K:\bbX\times \bbX\rightarrow \bbC$ is said to be \emph{determinantal}, denoted as $\calX\sim\DPP(K)$, if its $k$-correlation functions satisfy
	  \begin{equation*}
	  \label{eq:k-correlation_function_DPP}
	    \rho_k(x_1,\dots,x_k)
	      = \det \lrb{K(x_i, x_j)}_{i,j=1}^k,
	    \quad \forall k\geq 1.
	  \end{equation*}
		Most DPPs in ML correspond to the finite case where $\bbX = \lrcb{1,\dots,M}$ and the reference measure is $\mu=\sum_{i=1}^M \delta_i$ \citep{KuTa12}.
		In this context, the kernel function becomes an $M\times M$ matrix $\bfK$, and the correlation functions refer to inclusion probabilities.
	DPPs are thus often defined in ML as $\calX\sim \DPP(\bfK)$ if
	  \begin{equation}
	  \label{eq:inclusion_proba_finite}
	    \Proba{S \subset \calX} = \det {\bfK}_S,
	      \quad\forall S\subset \bbX,
	  \end{equation}
	where ${\bfK}_S$ denotes the submatrix of $\bfK$ formed by the rows and columns indexed by $S$.
	If we further assume that the kernel matrix $\bfK$ is Hermitian, then the existence and unicity of the DPP in \Eqref{eq:inclusion_proba_finite} is equivalent to the condition that the spectrum of $\bfK$ lies in $[0,1]$.
	The result also holds for general Hermitian kernel functions $K$ with additional assumptions \cite[Theorem 3]{Sos00}.
	We note that there are also DPPs with nonsymmetric kernels \citep{BoDiFu10}.

	The main interest in DPPs in ML is that they can model diversity while being tractable.
	To see how diversity shows, we simply observe that \Eqref{eq:inclusion_proba_finite} entails
	\begin{equation*}
	\label{eq:2point_correlation_function_inclusion_proba_finite_case}
	  \Proba{\{i,j\} \subset \calX} 
	    = {\bfK}_{ii}{\bfK}_{jj}-{\bfK}_{ij}{\bfK}_{ji} 
	    = \Proba{\{i\} \subset \calX}
	      \times \Proba{\{i\} \subset \calX} 
	        - {\bfK}_{ij}{\bfK}_{ji}.
	\end{equation*}
	If $\bfK$ is Hermitian, the quantity ${\bfK}_{ij}{\bfK}_{ji} = \vert{\bfK}_{ij}\vert^2$ encodes how less often $i$ and $j$ should co-occur, compared to independent sampling with the same marginals.
	Most point processes that encode diversity are not tractable, in the sense that we do not have efficient algorithms to sample, marginalize, or compute normalization constants.
	DPPs are amenable to these tasks \citep{KuTa12}.

	% subsection definition (end)
	\subsection{Sampling} % (fold)
	\label{sub:sampling}

	  We assume henceforth that $K$ is Hermitian and satisfies suitable conditions \cite[Theorem 3]{Sos00} so that its spectral decomposition is available
	  \begin{equation*}
	  \label{eq:eigdec_kernel}
	    K(x,y)
	    \triangleq
	      \suml_{i=1}^{\infty}
	        \lambda_i \phi_i(x) \overline{\phi_i(y)},
	      \quad \text{with }
	        \int \phi_i(x) \overline{\phi_j(x)} \mu(\diff x) = \delta_{ij}.
	  \end{equation*}
	  In the \href{https://dppy.readthedocs.io/en/latest/finite_dpps/exact_sampling.html}{\textcolor{magenta}{finite case}}, the eigendecomposition of a Hermitian kernel $\bfK$ can always be computed.
	  \citet[Theorem 7]{HKPV06} proved that sampling $\DPP(K)$ can be done in two steps:
	  \begin{enumerate}
	    \item draw $B_i\sim\Ber(\lambda_i)$ independently and denote $\lrcb{i_1,\dots,i_{N}} = \lrcb{i:B_i=1}$,
	    \item sample from the DPP with kernel $\tilde{K}(x,y) = \sum_{n=1}^{N}\phi_{i_n}(x) \overline{\phi_{i_n}(y)}$.
	  \end{enumerate}
	  In particular, this shows that all DPPs are mixtures of \emph{projection} DPPs, that is, DPPs parametrized by an orthogonal projection kernel.
	  The second step of the sampling scheme can be performed using Algorithm 18 of \cite{HKPV06}, which we now describe.

	  \cite{HKPV06} first prove that a projection $\DPP(\tilde{K})$ generates configurations of $N=\Tr \tilde{K}$ points almost surely, and therefore, it is enough to sample $(X_1,\dots,X_N)$ with distribution
	  \begin{equation}
	  \label{eq:joint_distribution1}
	    \frac{1}{N!}
	      \det \lrb{\tilde K(x_m,x_n)}_{m,n=1}^N
	        \prod_{n=1}^N\diff \mu(x_n)
	    = \frac{1}{N!}
	        \Vol^2\lrcb{\Phi(x_1),\dots,\Phi(x_n)}
	          \prod_{n=1}^N \diff\mu(x_n),
	  \end{equation}
	  where $\Phi(x) \triangleq \lrp{\phi_{i_1}(x),\dots,\phi_{i_N}(x)}$ denotes the \emph{feature vector} associated with $x\in\bbX$, so that $\tilde{K}(x,y) = \Phi(y)^{\dagger} \Phi(x)$.
	  The algorithm boils down to expressing the chain rule for \Eqref{eq:joint_distribution1} as
	  \begin{equation}
	  \label{eq:joint_distribution2}
	    \underbrace{
	      \frac{1}{N}
	        \lrnorm{\phi(x_1)}^2
	          \diff\mu(x_1)
	      }_{p(x_1)}
	    \,\prod_{n=2}^{N}\,
	    \underbrace{
	      \frac{1}{N-(n-1)}
	        \lrnorm{ \Pi_{H_{n-1}^{\perp}} \phi(x_n)}^2\!\!
	          \diff\mu(x_n)
	      }_{p(x_n\vert x_1,\dots,x_{n-1})},
	  \end{equation}
	  where $\Pi_{H_{n-1}}$ is the orthogonal projection onto $H_{n-1} \triangleq \Span\lrcb{\phi(x_1), \dots, \phi(x_{n-1}) }$.
	  The crux of \citet[Algorithm 18]{HKPV06} is that since $\tilde K$ is a projection kernel, each term in \Eqref{eq:joint_distribution2} is a well-normalized conditional density, making \Eqref{eq:joint_distribution2} a chain rule.
	  It is then enough to sample from each conditional sequentially.
	  % \begin{equation}
	  % \label{eq:conditionals_densities}
	  %   p(x_1)
	  %     = \frac{1}{N} \lrnorm{\phi(x_1)}^2
	  %     \quad\text{and}\quad
	  %   p(x_n | x_1,\dots,x_{n-1})
	  %     = \frac{1}{N-(n-1)} \lrnorm{ \Pi_{H_{n-1}^{\perp}} \phi(x_n)}^2\!\!,
	  % \end{equation}
	  % \begin{align*}
	  %   g_1(x)           &= \frac{1}{N} \lrnorm{\phi(x)}^2
	  %                    &= \frac{1}{N} K(x,x) \\
	  %   g_{n | 1:n-1}(x) &= \frac{1}{N-(n-1)}
	  %                       \lrnorm{ \Pi_{H_{n-1}^{\perp}} \phi(x)}^2 \\
	  %                    &= \frac{1}{N-(n-1)}
	  %                       \lrb{K(x,x) - \overline{K(x,x_{1:n-1})} \lrb{K(x_k,x_l)}_{k,l=1}^{n-1} K(x_{1:n-1},x)}
	  % \end{align*}

	  We make few remarks.
	  First, each conditional in \Eqref{eq:joint_distribution2} can be expressed as a ratio of two determinants and further expanded with Woodbury's formula.
	  In that case, in each conditional, ML practitioners will recognize  a posterior variance in Gaussian process regression with kernel $\tilde K$ \cite[Equation 2.26]{RaWi06}.
	  We insist that this link with Gaussian processes is a \href{https://dppy.readthedocs.io/en/latest/finite_dpps/exact_sampling.html#caution}{\textcolor{magenta}{consequence}} of $\tilde K$ being a \emph{projection} kernel.
	  Second, we observe that the chain rule in \Eqref{eq:joint_distribution2} has a strong Gram-Schmidt flavor since it actually comes from a recursive application of the base$\times$height formula.
	  In the end, DPPs favor configuration of points whose feature vectors $\Phi(x_1),\dots, \Phi(x_N)$ span a large volume, which is another way of understanding repulsiveness.

	  The previous sampling scheme is exact and generic but requires the eigendecomposition of the underlying kernel.
	  In the finite setting, this corresponds to an initial $\calO(M^3)$ cost, but then the procedure has an average cost of $\calO(M\lrb{\Tr \bfK}^2)$
	  % which may still be problematic for large $M$,
	  \citep{TrBaAm18}.
	  In the continuous case, there is an additional cost of the rejection sampling routine required for sampling each conditional in \Eqref{eq:joint_distribution2}.
	  In applications where these costs are a bottleneck, users rely on approximate sampling.
	  Research has focused on two main directions: kernel approximation \citep{AKFT13} and MCMC samplers \citep{AnGhRe16, LiJeSr16c, GaBaVa17}.

	  To conclude, some specific DPPs admit more efficient exact samplers that do not rely on \Eqref{eq:joint_distribution2}, \eg uniform spanning trees \citep{PrWi98} or eigenvalues of random matrices.
	  For instance, a $\beta$-ensemble is a set of $N$ points $\{X_1,\dots,X_N\}\subset \bbR$ with pdf
	  \begin{equation*}
	  \label{eq:beta_ensemble_pdf}
	    \frac{1}{Z_{N,\beta}}
	    \prod_{m< n}
	      \lrabs{x_m-x_n}^{\beta}
	    \prod_{i= 1}^N
	      \omega(x_i),
	  \end{equation*}
	  where $\beta>0$.
	  For some choices of the weight function $\omega$, the $\beta$-ensemble can be sampled as the \href{https://dppy.readthedocs.io/en/latest/continuous_dpps/beta_ensembles.banded_models.html}{\textcolor{magenta}{spectrum of simple tridiagonal random matrices}} \citep{DuEd02,KiNe04}.
	  When $\beta=2$, $\beta$-ensembles are projection DPPs, and therefore examples of continuous DPPs that can be sampled in $\calO(N^2)$.
	  Some of these ensembles are of direct interest to MLers.
	  For instance, the \emph{Laguerre} ensemble, where $\omega$ is a Gamma pdf, is the distribution of the eigenvalues of the empirical covariance matrix of \iid Gaussian vectors.

	% subsection sampling (end)

% section determinantal_point_processes (end)

\section{The \DPPy\ toolbox} % (fold)
\label{sec:the_dppy_toolbox}

  \DPPy\ handles objects that fit the natural definition of the different DPPs models.
  \begin{itemize}
	  \item The \DPPy\ object corresponding to the finite $\DPP(\bfK)$ can be instantiated as
    \begin{nscenter}
      \texttt{FiniteDPP(%
        kernel\pyus type\pyeq\pystr{inclusion},\,%
        projection\pyeq\pygreen{False},\,%
        \pykwargs\pylrcb{%
          \pystr{K}:K}%
        ).}
    \end{nscenter}
		It has two main sampling methods, namely \texttt{.sample\pyus exact()} and \texttt{.sample\pyus mcmc()}, implementing different variants of the exact sampling scheme and current state-of-the-art MCMC samplers.

		\item The \DPPy\ object corresponding to $\beta$-ensembles can be instantiated as
    \begin{nscenter}
      \texttt{BetaEnsemble(%
        ensemble\pyus name\pyeq\pystr{laguerre},\,%
        beta\pyeq\pygreen{3.14}%
        )}.
    \end{nscenter}
		It has one sampling method
    \begin{nscenter}
      \texttt{.sample(%
        sampling\pyus mode\pyeq\pystr{banded},\,%
        \pykwargs\pylrcb{%
          \pystr{shape}:\pygreen{10},\,%
          \pystr{scale}:\pygreen{2.0},\,%
          \pystr{size}:\pygreen{50}}%
        )}
    \end{nscenter}
		and two methods for display: \texttt{.plot()} to plot the last realization and \texttt{.hist()} to construct the empirical distribution.
  \end{itemize}
  More information can be found in the documentation\footnoteref{fn:docs} and the corresponding Jupyter \href{https://github.com/guilgautier/DPPy/tree/master/notebooks}{\textcolor{magenta}{notebooks}}, which showcase \DPPy\ objects.

% section the_dppy_toolbox (end)

\section{Conclusion and future work} % (fold)
\label{sec:conclusion_and_future_work}

	\DPPy\ can readily serve as research and teaching material.
  \DPPy\ is also ready for other contributors to add content and enlarge its scope, \eg with procedures for learning kernels.

% section conclusion_and_future_work (end)

% Acknowledgements should go at the end, before appendices and references
%\newpage
%{\small
\newpage
\acks{%
We would like to thank \href{https://guillep.github.io/}{Guillermo Polito} for leading our reproducible research \href{https://github.com/CRIStAL-PADR/reproducible-research-SE-notes}{workgroup}, this project owes him a lot.
We acknowledge funding by European CHIST-ERA project DELTA, the French Ministry of Higher Education and Research, the Nord-Pas-de-Calais Regional Council, Inria and Otto-von-Guericke-Universit\"at Magdeburg associated-team north-European project Allocate, and French National Research Agency projects ExTra-Learn (n.ANR-14-CE24-0010-01) and BoB (n.ANR-16-CE23-0003).
}

% Manual newpage inserted to improve the layout of sample file - not
% needed in general before the appendices/bibliography.

% \newpage

% \appendix
% \section*{Appendix A.}
% \label{app:theorem}

% Note: in this sample, the section number is hard-coded in. Following
% proper LaTeX conventions, it should properly be coded as a reference:

%In this appendix we prove the following theorem from
%Section~\ref{sec:textree-generalization}:

% \vskip 0.2in

\bibliography{biblio_new}

\end{document}
