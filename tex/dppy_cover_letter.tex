\documentclass[twoside,11pt]{article}

% Any additional packages needed should be included after jmlr2e.
% Note that jmlr2e.sty includes epsfig, amssymb, natbib and graphicx,
% and defines many common macros, such as 'proof' and 'example'.
%
% It also sets the bibliographystyle to plainnat; for more information on
% natbib citation styles, see the natbib documentation, a copy of which
% is archived at http://www.jmlr.org/format/natbib.pdf

\usepackage{jmlr2ecover}
\usepackage{fontawesome} % For fancy icons
\usepackage{hyperref}
\hypersetup{
  colorlinks=true,
  linkcolor=blue,
  urlcolor=gray,
  citecolor=blue
}
\usepackage[utf8x]{inputenc}
% \usepackage{natbib}
\usepackage{xcolor}
% Specific to this .tex
\newcommand{\DPPy}{\textsf{DPPy}}
% TODO command
\newcommand*{\todo}[1]{\textcolor{red}{TODO:{#1}}}
\newcommand*{\anstodo}[1]{\textcolor{blue}{ANS-TODO:{#1}}}

% Footnotes 1, 2, 3 and then fancy ones!
\makeatletter
\@addtoreset{footnote}{page}
\makeatother

\renewcommand{\thefootnote}{\ifcase\value{footnote}\or\color{black}{1}\or\color{black}{2}\or\color{black}{3}\or\color{black}{\faGithub}\or\color{black}{\faBook}\or\color{black}{\faGears}\or\color{black}{\faNewspaperO}\fi} 

% Cross ref a foonote
\makeatletter
\newcommand\footnoteref[1]{\protected@xdef\@thefnmark{\ref{#1}}\@footnotemark}
\makeatother

\addtolength{\skip\footins}{-3pc plus 5pt}

% Heading arguments are {volume}{year}{pages}{date submitted}{date published}{paper id}{author-full-names}

% \jmlrheading{xx}{2018}{xx-xx}{9/18}{xx/xx}{gabava18}{Guillaume Gautier, R\'emi Bardenet, and Michal Valko}

% Short headings should be running head and authors last names

% \ShortHeadings{\DPPy}{Gautier, Bardenet, and Valko}

\begin{document}

\title{\emph{Cover letter}\\[1em]
\DPPy: Sampling Determinantal Point Processes with Python}

\author{\name Guillaume Gautier \email g.gautier@inria.fr \\
       \name R\'emi Bardenet \email remi.bardenet@gmail.com \\
       \name Michal Valko \email michal.valko@inria.fr\\
}

% \editor{}

\maketitle

\vspace{2em}

\pagenumbering{gobble}
\setcounter{footnote}{3}

Dear Editorial Board,\\

We submit the \DPPy\ Python library to the machine learning open source software (MLOSS) section of the Journal of Machine Learning Research.
\DPPy\ is the acronym for Determinantal Point Processes (DPPs) with Python.\\

\DPPy\ tackles the nontrivial task of sampling DPPs with a turnkey implementation of all exact and approximate algorithms so far to sample finite DPPs.
We also provide a few algorithms for non-stationary continuous DPPs that are related to random projections and random covariance matrices, which may raise the interest of MLers.\\

The project is hosted on GitHub\footnote{\url{https://github.com/guilgautier/DPPy}} and supported by an extensive documentation\footnote{\url{https://dppy.readthedocs.io}} which provides the essential mathematical background and illustrates some key properties of DPPs through \DPPy\ objects and their methods.
We use Travis\footnote{\url{https://travis-ci.com/guilgautier/DPPy}} for continuous integration and to certify that \DPPy\ runs on Linux and MacOS plateforms with Python 3.4+.\\

In line with the reproducibily purpose, \DPPy\ does not rely on any proprietary software and is released under the MIT license.
Moreover, the companion paper\footnote{\url{https://github.com/guilgautier/DPPy_paper}} together with the present cover letter are also available on GitHub and fully reproducible.\\

We declare that this work is neither published nor submitted to any journal or conference.
All authors agree to the reviewing process of the Journal of Machine Learning Research.
We declare no conflict of interest whatsoever.\\

Thank you for your consideration.\\

Mr. Guillaume Gautier (Ph.D. student)\\
\indent Dr. Rémi Bardenet (Research scientist)\\
\indent Dr. Michal Valko (Research scientist)\\

Version to be reviewed: \DPPy\ \href{https://github.com/guilgautier/DPPy/tags}{v0.1.0}
\todo{(gg) tag final commit of DPPy to create a release through Travis}

\end{document}