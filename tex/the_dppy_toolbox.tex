% !TEX root = dppy_paper.tex 
\section{The \DPPy\ toolbox} % (fold)
\label{sec:the_dppy_toolbox}

  \DPPy\ handles objects that fit the natural definition of the different DPPs models.
  \begin{itemize}
	  \item The \DPPy\ object corresponding to the finite $\DPP(\bfK)$ can be instantiated as
    \begin{nscenter}
      \texttt{FiniteDPP(%
        kernel\pyus type\pyeq\pystr{inclusion},\,%
        projection\pyeq\pygreen{False},\,%
        \pykwargs\pylrcb{%
          \pystr{K}:K}%
        ).}
    \end{nscenter}
		It has two main sampling methods, namely \texttt{.sample\pyus exact()} and \texttt{.sample\pyus mcmc()}, implementing different variants of the exact sampling scheme and current state-of-the-art MCMC samplers.

		\item The \DPPy\ object corresponding to $\beta$-ensembles can be instantiated as
    \begin{nscenter}
      \texttt{BetaEnsemble(%
        ensemble\pyus name\pyeq\pystr{laguerre},\,%
        beta\pyeq\pygreen{3.14}%
        )}.
    \end{nscenter}
		It has one sampling method
    \begin{nscenter}
      \texttt{.sample(%
        sampling\pyus mode\pyeq\pystr{banded},\,%
        \pykwargs\pylrcb{%
          \pystr{shape}:\pygreen{10},\,%
          \pystr{scale}:\pygreen{2.0},\,%
          \pystr{size}:\pygreen{50}}%
        )}
    \end{nscenter}
		and two methods for display: \texttt{.plot()} to plot the last realization and \texttt{.hist()} to construct the empirical distribution.
  \end{itemize}
  More information can be found in the documentation\footnoteref{fn:docs} and the corresponding Jupyter \href{https://github.com/guilgautier/DPPy/tree/master/notebooks}{\textcolor{magenta}{notebooks}}, which showcase \DPPy\ objects.

% section the_dppy_toolbox (end)